\section{Performance Analysis}
Having validated it's accuracy, the analytical model is now used to investigate the performance of SDN and NFV enabled networks under load.

Analysis of the average arrival rate to each node shows that the edge switches typically receive the most traffic, followed by the aggregate and core switches. Under heavy loads the edge switches will be overwhelmed first resulting in high latency or packet loss. \pref{fig:length_chain} shows how this issue is aggravated by the addition of NFV features into the network. All switches receive proportionally the same increase in traffic but as aggregate switches are already the most loaded, they fail whilst other network elements have a non-negligible amount of spare capacity. \ref{fig:filtering_vnfs} shows that effective filtering network functions can reduce the load on the network. Optimisation of the ordering of filtering VNFs considering dependancies could help reduce the issues caused by increased lifetimes of packets in the network.

In \ref{fig:sdn_perc} we see how increasing the proportion of traffic that visits the SDN causes the network to become unstable at lower arrival rates. Analysis of the average arrival rate at servers shows that whilst they typically experience lower load than the network switches, they can receive substantially more traffic (from \ref{eq:arr_srv}, an additional $k_{vsw} \cdot \lambda * p_{SDN}$) when a large proportion of the packets are required to visit the root SDN controller. If the virtual switch is installed on a server that is able to handle the same rate of traffic as the other switches the edge switch will still be overloaded first. However if the server is substantially less powerful, as in these tests, the servers will be overwhelmed first.