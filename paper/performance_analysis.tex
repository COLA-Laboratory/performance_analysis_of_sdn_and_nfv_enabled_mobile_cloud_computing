\section{Performance Analysis}
\label{sec:performance}
Having validated its accuracy, the analytical model can now be used to investigate the performance of SDN and NFV enabled networks under load. In particular it is useful to be able to identify potential bottlenecks that would lead to poor performance.

Analysis of the average arrival rate to each node shows that the edge switches typically receive the most traffic, followed by the aggregate and core switches. Under heavy loads the edge switches will be the first network components to receive a higher arrival rate than their service rate leading to high latency or packet loss. Figure \ref{fig:length_chain} shows that this issue is aggravated by the addition of NFV features into the network. With longer service chains, although all switches receive proportionally the same increase in traffic, aggregate switches are already the most loaded so they can be overloaded whilst other network elements have spare capacity.

In Figure \ref{fig:sdn_perc} we see that increasing the proportion of traffic that visits the SDN causes the network to become unstable at lower arrival rates. Analysis of the average arrival rate at servers shows that whilst they typically experience lower load than the network switches, they can receive substantially more traffic (from Equation \ref{eq:arr_srv}, an additional $k_{vsw} \cdot p_{sdn} \cdot \lambda$) when a large proportion of the packets are required to visit the root SDN controller. If the virtual switches are able to handle the same rate of traffic as the physical switches the edge switch remains as the bottleneck. However if the virtual switches are substantially less powerful, as in these tests, the virtual switches can become the bottleneck.