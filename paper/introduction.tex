% !TeX root=main.tex

\section{Introduction}
\label{sec:introduction}
Emerging services such as Augmented and Virtual Reality, 4K video, and the Internet of Things will require incredible amounts of computational resources\cite{AndrewsBCHLSZ14}. At the heart of many of these new use cases is the datacentre, providing the required volumes of processing, storage and networking resources. The traditional approach of 'scaling-up' a datacentre: acquiring more powerful yet more expensive equipment to meet demand is no longer tenable \cite{VahdatAFMPR10}. Faced with high capital and operating expenditure, service providers have been seeking technologies that allow for more efficient usage of available resources and simplify management of new and existing equipment. Increasingly, the solution to these problems has been virtualisation \cite{HeddeghemLLCPD14} and modern datacentres have embraced the concepts of network function virtualisation (NFV) and software defined networking (SDN).

Modern datacentres require components capable of providing functions such as load balancing, firewalls and intrusion detection systems. Traditionally these network functions would be provided by purpose engineered network hardware greatly hindering the network innovation. In an NFV enabled network, virtual network functions (VNFs) are run on virtual machines on commodity hardware. These VNFs can be moved, scaled or destroyed on demand, allowing for efficient placement and allocation of resources and significantly accelerating the deployment of new services.

Datacentres contain large interconnection networks that allow communication between servers. Software Defined Networking (SDN) allows for dynamic configuration of this network and the other datacentre components \cite{KimF13,HaresW13}. A logically centralised SDN controller maintains a global view of the network. The controller provides instructions that describe how packets should be routed through the network to 'dumb' switches. This centralises the networks intelligence, simplifying management and allowing for new and complex networking structures.

SDN and NFV are often considered complementary technologies \cite{MatiasGTUJ15}; with the flexible placement enabled by NFV and the complex routing permitted by SDN complex and dynamic networks can be created. Despite this, existing research in modelling of both technologies has typically considered them in isolation.

Many methods of modelling SDN alone are available in the literature. Longo et al. \cite{LongoDBS15} proposed a model of the reliability of a two layer hierarchical SDN controller. Azodolmolky et al. \cite{AzodolmolkyWY13} also examine the two layer SDN controller but use network calculus to determine the worst case delay and the minimum buffer size required to prevent packet loss. Wang et al. \cite{MiaoMWWH16} developed a more realistic SDN model by considering the bursty and correlated arrivals of packets and a high and low priority queue at an SDN enabled switch. These models focus solely on SDN, ignoring the particular interactions between SDN and the network it would be deployed on.

As with SDN, NFV modelling has similarily had a narrow focus. Prados-Garzon et al. \cite{Prados-GarzonAR17} produced a detailed model of a single VNF which is composed of several VNF components and calculated the average response time of the VNF. Gebert et al. \cite{GebertZLST16} analysed a single VNF in detail, modelling each queue in the packet processing pipeline of a Linux x86 system. To the best of our knowledge, only Fahmin et al. \cite{FahminLHLS17} have considered both NFV and SDN, they modelled the performance of two methods of combining SDN and NFV in the network. However they consider a simplified network with only one switch and one VNF.

To fill this gap, a comprehensive analytical model is developed in this work to investigate the performance of a datacentre network in the presence of multiple NFV services and a virtualised SDN implementation. To consider the practical datacenter system, the proposed model is designed with the aim of capturing the dynamicity of NFV service deployments, for instance multiple NFV service chains share the same physical network and each NFV chain could contain different numbers of VNFs. The end-to-end latency is derived based on the developed model. To validate the accuracy of the developed analytical model, extensive simulation experiments are conducted under various network configurations. To illustrate its applications, the analytical model is then used as an efficient evaluation tool to analyse performance bottlenecks of datacentre networks. 

The remainder of this paper is organised as follows. Section \ref{sec:preliminaries} discusses the details of the network architecture that is modelled in this work. In Section \ref{sec:analytical_model} we derive the analytical model for the network. Section \ref{sec:validation} validates the accuracy of this model with extensive simulation experiments. Section \ref{sec:performance} explores the implications of the model and Section \ref{sec:conclusions} concludes the paper and examines future research directions. 