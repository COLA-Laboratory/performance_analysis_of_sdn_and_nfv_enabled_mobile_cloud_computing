% !TeX root=main.tex

\section{Introduction}
\label{sec:introduction}
Emerging services such as Augmented and Virtual Reality, 4K video, and the Internet of Things will require incredible amounts of computational resources\cite{AndrewsBCHLSZ14}. At the heart of many of these new use cases is the datacentre, providing the required volumes of processing, storage and networking resources. The traditional approach of 'scaling-up' a datacentre: acquiring more powerful yet more expensive equipment to meet demand is no longer tenable \cite{VahdatAFMPR10}. Faced with high capital and operating expenditure, service providers have been seeking technologies that allow for more efficient usage of available resources and simplify management of new and existing equipment. Increasingly, the solution to these problems has been virtualisation \cite{HeddeghemLLCPD14} and modern datacentres have embraced the concepts of network function virtualisation (NFV) and software defined networking (SDN).

Modern datacentres require components capable of providing functions such as load balancing, firewalls and intrusion detection systems. Traditionally these network functions would be provided by purpose engineered network hardware greatly hindering the network innovation. In an NFV enabled network, network functions are instead run on virtual machines on commodity hardware. These Virtual Network Functions (VNFs) can be moved, scaled or destroyed on demand, allowing for efficient placement and allocation of resources, significantly accelerating the deployment of new services.

Datacentres contain large interconnection networks that allow communication between servers. Software Defined Networking (SDN) allows for dynamic configuration of this network and the other datacentre components \cite{KimF13,HaresW13}. A logically centralised SDN controller maintains a global view of the network. The controller provides instructions that describe how packets should be routed through the network to 'dumb' switches. This centralises the networks intelligence, simplifying management and allowing for complex and flexible networking structures.

SDN and NFV are often considered complementary technologies~\cite{MatiasGTUJ15}; with NFV freeing services from particular servers and SDN separating services from switches, rapid deployment and configuration of services is possible. Unfortunately, whilst powerful technologies their great flexibility adds additional complexity to the design of data centres. 

Analytical models can provide insight into datacentre design by formally defining the interactions between key parameters of the design such as the size of the datacentre, the supported services and the required performance. As SDN and NFV are complementary technologies and may often be deployed together, it is important to identify how their interactions can affect the performance of the datacentre. However existing research in modelling of both technologies has typically considered them in isolation.

Many methods of modelling SDN alone are available in the literature. Longo et al. \cite{LongoDBS15} proposed a model of the reliability of a two layer hierarchical SDN controller. Azodolmolky et al. \cite{AzodolmolkyWY13} also examine the two layer SDN controller but use network calculus to determine the worst case delay and the minimum buffer size required to prevent packet loss. Wang et al. \cite{MiaoMWWH16} developed a more realistic SDN model by considering the bursty and correlated arrivals of packets and a high and low priority queue at an SDN enabled switch. These models focus solely on SDN, ignoring the particular interactions between SDN and the network it would be deployed on.

Research on NFV modelling has also had a narrow focus. Prados-Garzon et al. \cite{Prados-GarzonAR17} produced a detailed model of a single VNF which is composed of several VNF components and calculated the average response time of the VNF. Gebert et al. \cite{GebertZLST16} analysed a single VNF in detail, modelling each queue in the packet processing pipeline of a Linux x86 system. To the best of our knowledge, only Fahmin et al. \cite{FahminLHLS17} have considered both NFV and SDN. However they consider a simplified network with only one switch and one VNF.

In this work a comprehensive analytical model is developed to investigate the performance of a datacentre network in the presence of multiple NFV services and a virtualised SDN implementation. The impact of multiple NFV service chains of varying lengths coexisting on the same physical network is considered as are interactions with the SDN controller.

The remainder of this paper is organised as follows. Section \ref{sec:preliminaries} discusses the details of the network architecture that is modelled in this work. In Section \ref{sec:analytical_model} we derive the analytical model for the network. Section \ref{sec:validation} validates the accuracy of this model with extensive simulation experiments. Finally Section \ref{sec:conclusions} concludes the paper. 