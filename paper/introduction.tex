% !TeX root=main.tex

\section{Introduction}
\label{sec:introduction}

Emerging mobile services such as Virtual Reality, 4K video, and tactile internet, consume incredible amounts of compute, storage and bandwidth resources\cite{AndrewsBCHLSZ14}. However, due to the inherent constraints of their size, weight and power, mobile devices become struggle to meet the strict resource requirements of new applications. Mobile Cloud Computing (MCC) \cite{7809047} \cite{7364240} has been considered as a key technology for mobile devices to address this issue. Through migrating the local applications and service to a mobile cloud datacenter, MCC brings mobile devices the benefits of extending the battery life, increasing the data storage and enhancing the processing power. To realise this ambition, Software Defined Networking (SDN) and Network Function Virtualisation (NFV) have been regarded as two promising and complementary technologies in MCC datacenter to simplify the datacenter network management and improve the resource utilisation and service flexibility. 

SDN is a new networking architecture that can simplify the network management and accelerate network innovation. It is implemented by decoupling the network control from the underlying network infrastructure and creating a software-programmable controller. A logically centralised SDN controller maintains a global view of the network, helping the network operator to design network services and determines how packets should be routed through the network \cite{KimF13} \cite{HaresW13}. With the centralised network control, networks intelligence is migrated from the underlying network devices to SDN controller, enabling network operators to manage the entire network consistently and holistically. 

NFV is a novel network architecture which allows for flexibility in service provisioning. Traditionally, services are constructed by connecting chains of purpose built computers each performing a particular function. These may be traditional data centre functions such as firewalls and load balancers, or mobile communications functions such as the Packet and Service gateways in the 4G Evolved Packet Core. NFV decouples these functions from the hardware by implementing these network functions in software on virtual machines (VMs). These Virtual Network Functions (VNFs) can be moved, scaled or destroyed on demand, allowing for efficient placement and allocation of resources, significantly accelerating the deployment of new services. 

SDN and NFV are often considered complementary technologies in practical deployments \cite{MatiasGTUJ15}.  For instance, when a new MCC service needs to be deployed in the cloud datacenter, the cloud management system firstly design a service chain and leverages Virtual Network Function (VNF) manager to deploy VNFs in the underlying Virtual Machines (VMs) or containers. After initiating the VNFs, the address of the VMs or containers as well as the NFV chain will be sent to SDN controller, which is responsible for establishing the connections between difference VNFs and collecting the stochastic information cloud management system for service optimisation. From the above example, it can be seen that SDN plays an important role in the deployment, management and optimisation of overall lifecycle of NFV service provisioning. Therefore, it is necessary to jointly consider SDN and NFV in the datacenter network management and service provisioning. For the system optimisation, analytical models can provide insight of system operation by formally defining the interactions among key parameters such as the scale and resources utilisation of the datacenter, the traffic generated by the end devices and the Quality of Service (QoS) performance that should be satisfied. There have been some research efforts to analyse the performance of SDN and NFV network architecture \cite{LongoDBS15} \cite{MiaoMWWH16} \cite{Prados-GarzonAR17} \cite{GebertZLST16} \cite{8624508}. For instance, for modelling the performance of SDN networks, Longo et al. \cite{LongoDBS15} proposed a model to investigate the reliability performance of SDN network, which is based on a two layer network management architecture. Azodolmolky et al. \cite{AzodolmolkyWY13} exploited network calculus to investigate the  worse case delay performance of SDN network as well as minimum buffer size required for a given delay constraint. To capture the burstness of the network traffic, Wang et al. \cite{MiaoMWWH16} developed an analytical model to investigate an SDN architecture with the traffic following the Markov-modulated Poisson Process (MMPP). A high and low priority queue model was proposed in this work to capture the features of the practical SDN deployment. For analysing the performance of NFV architecture, Prados-Garzon et al. \cite{Prados-GarzonAR17} designed an analytical model to investigate the average response time of a single NFV service provisioning. Gebert et al. \cite{GebertZLST16} modeled each step in the packet processing pipeline of a Linux x86 system, which is based on single VNF deployment scenario. To analyse the stochastic performance of multiple VNF scenario, an analytical model was proposed in \cite{8624508} to exploit the stochastic network calculus to investigate of the end-to-end performance of an NFV service provisioning, which could obtain the worse case of network transmission for a given QoS requirements. Although, these existing works provide some insights into the performance of SDN and NFV network architecture in various network scenarios, they seldom jointly consider these two technologies in the performance analysis. From the perspective of service deployment and provisioning, SDN and NFV are complementary technologies and always deployed together. Therefore, it is important to investigate the performance of network infrastructure with both SDN and NFV support, especially identifying how their interactions can affect the performance of service provisioning. To the best of our knowledge, only Fahmin et al. \cite{FahminLHLS17} have considered both NFV and SDN in their analytical model. However the network infrastructure adopted in \cite{FahminLHLS17} consists of only one switch and one VNF, which can be hardly to be applicable to a large scale datacenter network. In order to reap the benefits of SDN and NFV for MCC applications, there is an urgent need to develop a novel analytical model which can jointly consider two complementary technologies in a large-scale datacenter network. 
 
To fill this gap, a comprehensive analytical model is proposed in this work to investigate the performance of SDN and NFV enabled MCC datacenter networks. To capture the unique features of real-world SDN and NFV deployments a network architecture is firstly abstracted in this study with multiple NFV chains and a virtualised SDN implementation, where the SDN controller determines how traffic is routed among the VNFs. The analytical model is developed with the aim of understanding the interactions between SDN and NFV when they are deployed under the same underlying physical infrastructure, e.g the impact of the length of NFV service chain on the traffic engineering performance of SDN networks. The end-to-end performance in terms of average latency is obtained by the developed model and validated through extensive simulation experiments under different network configurations. In addition, the proposed analytical model could be used as an effective tool to optimise the design of services and networks in MCC datacenters.

The remainder of this paper is organised as follows. Section \ref{sec:preliminaries} discusses the details of the network architecture that is modelled in this work. In Section \ref{sec:analytical_model} we derive the analytical model for the network. Section \ref{sec:validation} validates the accuracy of this model with extensive simulation experiments. Finally Section \ref{sec:conclusions} concludes the paper.