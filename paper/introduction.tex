% !TeX root=main.tex

\section{Introduction}
\label{sec:introduction}

Emerging mobile services such as Augmented and Virtual Reality, 4K video, and the Internet of Things will require incredible amounts of compute, storage and bandwidth resources\cite{AndrewsBCHLSZ14}. Due to the inherent constraints of their size, weight and power, mobile devices struggle to meet the performance requirements of adanced applications such as compute-intensive operations like object recognition. Mobile Cloud Computing (MCC) \cite{7809047}, \cite{7364240} has been considered as a key technology for mobile devices to mitigate these resource constraints. In MCC, mobile devices off-load the local applications and service to a mobile cloud datacenter. However, with the explosive growth of mobile devices and increasingly resource-hungry applications being deployed in the datacenter, cloud service providers must efficiently use resources to balance high Capital Expenditure (CAPEX) and Operating Expenditure (OPEX) against revenue capability \cite{VahdatAFMPR10}. To keep pace, service providers have been seeking technologies that allow for more efficient usage of available resources and simplify management of new and existing equipment. Software Defined Networking (SDN) and Network Function Virtualisation (NFV) are two promising technologies that may revolutionise network design and operations.

SDN is a new networking architecture that can simplify the network management and accelerate network innovation. It is implemented by decoupling the network control from the underlying network infrastructure and creating a software-programmable controller. A logically centralised SDN controller maintains a global view of the network, helping the network operator to design network services and provides detailed instructions of how packets should be routed through the network \cite{KimF13,HaresW13}. This centralises the networks intelligence, enabling network operators to manage the entire network consistently and holistically. 

NFV is a novel network architecture which allows for flexibility in service provisioning. Traditional services are constructed by connecting chains of purpose built computers each performing a particular function, including traditional data centre functions such as firewallls and load balancers and Mobile Communications functions such as the Packet and Service gateways in the 4G Evolved Packet Core. NFV instead implements these network functions in software run on virtual machines (VMs). These Virtual Network Functions (VNFs) can be moved, scaled or destroyed on demand, allowing for efficient placement and allocation of resources, significantly accelerating the deployment of new services. 

SDN and NFV are often considered complementary technologies \cite{MatiasGTUJ15} in practical deployments. For instance, when a mobile service is initialised in the cloud datacenter, the cloud management system may establish a service for the mobile application by deploying several VNFs. Then the cloud management system can leverage SDN technology to build transmission paths to link the different VNFs to realise the service. 

Analytical models can provide insight into datacentre design by formally defining the interactions between key parameters of the design such as the size of the datacentre, the supported services and the required performance. There have been some research efforts to analyse the performance of SDN and NFV network architecture. For instance, for modelling SDN networks, Longo et al. \cite{LongoDBS15} proposed a model to investigate the reliability of a two layer hierarchical SDN controller. In order to determine the worst case delay and the minimum buffer size required for given QoS requirement, Azodolmolky et al. \cite{AzodolmolkyWY13} exploited network calculus to investigate the performance of a two layer SDN architecture. To capture the burstness of the network traffic in SDN network, Wang et al. \cite{MiaoMWWH16} developed an analytical model to investigate the SDN architecture with bursty and corelated traffic arrivals. A high and low priority queue model was proposed in this work to reflect the practical deployment of SDN networks. For analysing the performance of NFV architecture, Prados-Garzon et al. \cite{Prados-GarzonAR17} produced a detailed model of a single VNF which is composed of several VNFs and calculated the average response time of the VNF. Gebert et al. \cite{GebertZLST16} analysed a single VNF in detail, modelling each queue in the packet processing pipeline of a Linux x86 system. Although, these existing works provide some insights into the performance of SDN and NFV network architecture, they do not jointly consider these two technologies. As SDN and NFV are complementary technologies and are often deployed together, it is important to identify how their interactions can affect the performance of the datacentre. To the best of our knowledge, only Fahmin et al. \cite{FahminLHLS17} have considered both NFV and SDN in analytical model. However they consider a simplified network with only one switch and one VNF, which is too small to be applicable to a large scale network. In order to reap the benefits of SDN and NFV for MCC applications, there is an urgent need to develop a novel analytical model, which can jointly consider these two complementary technologies in a large-scale datacenter network. 

To fill this gap, a comprehensive analytical model is proposed in this work to investigate the performance of SDN and NFV enabled datacenter networks. To capture the unique features of real-world SDN and NFV deployments a network architecture is firstly abstracted in this study with multiple NFV chains and a virtualised SDN implementation, where the SDN controller determines how traffic is routed among the VNFs. The analytical model is developed with the aim of understanding the interactions between SDN and NFV when they are deployed under the same underlying physical infrastructure, e.g the impact of the length of NFV service chain on the traffic engineering performance of SDN networks. The end-to-end performance in terms of average latency is provided under different network configurations. In addition, the proposed analytical model could be used as an effective tool to optimise the design of services and networks when deploying networks in Mobile Cloud datacenters.

The remainder of this paper is organised as follows. Section \ref{sec:preliminaries} discusses the details of the network architecture that is modelled in this work. In Section \ref{sec:analytical_model} we derive the analytical model for the network. Section \ref{sec:validation} validates the accuracy of this model with extensive simulation experiments. Finally Section \ref{sec:conclusions} concludes the paper.