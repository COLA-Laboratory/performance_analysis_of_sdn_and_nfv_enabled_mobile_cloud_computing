% !TeX root=main.tex

\section{Preliminaries}
\label{sec:preliminaries}
This section first introduces the concepts of Network Function Virtualisation and Software Defined Networking and presents the implications of these changes to the communication network architecture.

\subsection{Network Function Virtualisation}
In telecommunications networks, services are composed of several network functions such as load balancers, firewalls and intrusion detection systems. Traditionally these network functions would be provided by specially engineered pieces of network hardware. In a Network Function Virtualisation (NFV) enabled network, these network functions are virtualised and run on virtual machines on commodity hardware.

\begin{figure}
\label{fig:dag_nfv}

\centering

\begin{tikzpicture}[
vnf/.style={rectangle, draw=black, fill=white, node distance=4mm and 14mm},]

\node[vnf] (S1_A) 				  {A};
\node[vnf] (S1_B) [right=of S1_A] {B};
\node[vnf] (S1_C) [right=of S1_B] {C};
\node[node distance = 4mm] [left=of S1_A]  {i)};

\node[vnf] (S2_A) [below= of S1_A] {A};
\node[vnf] (S2_B) [right= of S2_A] {B};
\node[vnf] (S2_C) [right= of S2_B] {C};
\node[vnf] (S2_D) [right= of S2_C] {D};
\node[node distance = 4mm] [left=of S2_A]  {ii)};

\draw[->] (S1_A.east)  -- node[above]{100\%}++(1.3,0) -- (S1_B.west);
\draw[->] (S1_B.east)  -- node[above]{100\%}++(1.3,0) -- (S1_C.west);

\draw[->] (S2_A.east)  -- node[above]{100\%}++(1.3,0) -- (S2_B.west);
\draw[->] (S2_B.east)  -- node[above]{50\%}++(1.3,0) -- (S2_C.west);
\draw[->] (S2_C.east)  -- node[above]{50\%}++(1.3,0) -- (S2_D.west);
\end{tikzpicture}

\caption{Two VNF service chains represented with directed acyclic graphs. Service i has a length of three and 100\% of the produced packets go through all VNFs. Service ii is four long and the proportion of packets passed between VNFs reduces at each VNF.}

\end{figure}

A service is a sequence of several virtual network functions (VNF). They can be defined using Directed Acyclic Graphs (DAG) which encapsulate dependencies between network functions. Some network functions such as firewalls can reduce the data rate by filtering out packets. The task of constructing a 'concrete' service from an abstract DAG is called service composition and has received interest in optimisation \cite{MehraghdamKK14, BariCAB15, WadaSYO12}. Examples of some concrete services can be seen in \pref{fig:dag_nfv}.

\subsection{Software Defined Networking}
Software Defined Networking (SDN) allow for dynamic routing of packets throughout the network and configuration of VNFs \cite{KimF13, HaresW13}. A logically centralised but often physically distributed SDN controller maintains a global view of the network. A portion of the packets generated in the network will visit the SDN controller.

In this work we model a two layer hierachical SDN controller similar to that evaluated by Azodolmolky et al. \cite{AzodolmolkyWY13}. In this architecture a local SDN controller exists at each server and an additional root SDN controller exists as an independant element. Arriving or produced packets are first evaluated at the local controller. If the local controller cannot process the packet it is sent to the root controller and then back to the server once resolved. A comparable implementation of this concept is VMwares NSX \cite{VMW18}.

\subsection{The Network Architecture}

In this work we extend the Fat Tree network topology \cite{Al-FaresLV08} to be SDN and NFV capable. \pref{fig:network_topology} illustrates this network. The network topology is defined as follows:

We define $k$ as the number of ports for each physical switch and $k_{vm}$ the number of ports for each virtual switch. There are $(k/2)^2$ core switches. Each core switch connects to one switch in each of $k$ pods. Each pod contains two layers (aggregation and edge) of $k/2$ switches. Each edge switch is connected to each of the $k/2$ aggregation switches of the pod. Each edge switch is connected to $k/2$ servers. Each server contains a virtual switch connected to $k_{vsw}$ VNFs. This topology results in $n=(k^3/4) \cdot k_{vsw}$ VNFs.

\begin{figure}

\label{fig:network_topology}

\centering

\begin{tikzpicture}[
every node/.style={node distance=6mm and 1mm},
server/.style={rectangle, draw=black, fill=black},
edge/.style={rectangle, draw=black, fill=black},
agg/.style={rectangle, draw=black, fill=black},
core/.style={rectangle, draw=black, fill=black},
sdn/.style={rectangle, draw=black, fill=black},
vm/.style={rectangle, draw=black, fill=black},
]

% Servers
\node[server]      (S1)                       {};
\node[server]      (S2)        [right=of S1]  {};
\node[server]      (S3)        [right=of S2]  {};
\node[server]      (S4)        [right=of S3]  {};
\node[server]      (S5)        [right=of S4]  {};
\node[server]      (S6)        [right=of S5]  {};
\node[server]      (S7)        [right=of S6]  {};
\node[server]      (S8)        [right=of S7]  {};
\node[server]      (S9)        [right=of S8]  {};
\node[server]      (S10)       [right=of S9]  {};
\node[server]      (S11)       [right=of S10] {};
\node[server]      (S12)       [right=of S11] {};
\node[server]      (S13)       [right=of S12] {};
\node[server]      (S14)       [right=of S13] {};
\node[server]      (S15)       [right=of S14] {};
\node[server]      (S16)       [right=of S15] {};

\node[text width=0.8cm] at (-2, 0)  {Servers};

% VMs
\node[vm]      (V1)        [below=of S1]  {};
\node[vm]      (V2)        [below=of S1, left=of V1]  {};
\node[vm]      (V3)        [below=of S1, right=of V1]  {};

\node[vm]      (V4)        [below=of S8]  {};
\node[vm]      (V5)        [below=of S8, left=of V4]  {};
\node[vm]      (V6)        [below=of S8, right=of V4]  {};

\node[vm]      (V7)        [below=of S15]  {};
\node[vm]      (V8)        [below=of S15, left=of V7]  {};
\node[vm]      (V9)        [below=of S15, right=of V7]  {};

\node[] at (-2, -.86)  {VNF};
\node [between=V1 and V4] {\large ...};
\node [between=V6 and V7] {\large ...};

% SDN Controller
\node[sdn]      (SC1)      	at (-0.3, 0.6) {};

% Edge/ToR
\node[edge]      (E1)        [above=of SC1,  between=S1 and S2]   {};
\node[edge]      (E2)        [above=of SC1,  between=S3 and S4]   {};
\node[edge]      (E3)        [above=of SC1,  between=S5 and S6]   {};
\node[edge]      (E4)        [above=of SC1,  between=S7 and S8]   {};
\node[edge]      (E5)        [above=of SC1,  between=S9 and S10]  {};
\node[edge]      (E6)        [above=of SC1, between=S11 and S12] {};
\node[edge]      (E7)        [above=of SC1, between=S13 and S14] {};
\node[edge]      (E8)        [above=of SC1, between=S15 and S16] {};

\node[text width=0.8cm] at (-2, .7)  {Edge};

% Aggregate
\node[agg]      (A1)        [above=of E1] {};
\node[agg]      (A2)        [above=of E2] {};
\node[agg]      (A3)        [above=of E3] {};
\node[agg]      (A4)        [above=of E4] {};
\node[agg]      (A5)        [above=of E5] {};
\node[agg]      (A6)        [above=of E6] {};
\node[agg]      (A7)        [above=of E7] {};
\node[agg]      (A8)        [above=of E8] {};

\node[text width=0.8cm] at (-2, 1.55)  {Aggregation};

% Core
\node[core]      (C1)        [above=of A1,  between=A1 and A2]  {};
\node[core]      (C2)        [above=of A3,  between=A3 and A4]  {};
\node[core]      (C3)        [above=of A5,  between=A5 and A6]  {};
\node[core]      (C4)        [above=of A7,  between=A7 and A8]  {};

\node[text width=0.8cm] at (-2, 2.35)  {Core};

%Lines
\draw[-] (V1.north)  -- (S1.south);
\draw[-] (V2.north)  -- (S1.south);
\draw[-] (V3.north)  -- (S1.south);

\draw[-] (V4.north)  -- (S8.south);
\draw[-] (V5.north)  -- (S8.south);
\draw[-] (V6.north)  -- (S8.south);

\draw[-] (V7.north)  -- (S15.south);
\draw[-] (V8.north)  -- (S15.south);
\draw[-] (V9.north)  -- (S15.south);

\draw[-] (S1.north)  -- (E1.south);
\draw[-] (S2.north)  -- (E1.south);
\draw[-] (S3.north)  -- (E2.south);
\draw[-] (S4.north)  -- (E2.south);
\draw[-] (S5.north)  -- (E3.south);
\draw[-] (S6.north)  -- (E3.south);
\draw[-] (S7.north)  -- (E4.south);
\draw[-] (S8.north)  -- (E4.south);
\draw[-] (S9.north)  -- (E5.south);
\draw[-] (S10.north) -- (E5.south);
\draw[-] (S11.north) -- (E6.south);
\draw[-] (S12.north) -- (E6.south);
\draw[-] (S13.north) -- (E7.south);
\draw[-] (S14.north) -- (E7.south);
\draw[-] (S15.north) -- (E8.south);
\draw[-] (S16.north) -- (E8.south);

\draw[-] (S1.north)  -- (SC1.south east);
\draw[-] (S2.north)  -- (SC1.south east);
\draw[-] (S3.north)  -- (SC1.south east);
\draw[-] (S4.north)  -- (SC1.south east);
\draw[-] (S5.north)  -- (SC1.south east);
\draw[-] (S6.north)  -- (SC1.south east);
\draw[-] (S7.north)  -- (SC1.south east);
\draw[-] (S8.north)  -- (SC1.south east);
\draw[-] (S9.north)  -- (SC1.south east);
\draw[-] (S10.north) -- (SC1.south east);
\draw[-] (S11.north) -- (SC1.south east);
\draw[-] (S12.north) -- (SC1.south east);
\draw[-] (S13.north) -- (SC1.south east);
\draw[-] (S14.north) -- (SC1.south east);
\draw[-] (S15.north) -- (SC1.south east);
\draw[-] (S16.north) -- (SC1.south east);

\draw[-] (E1.north)  -- (A1.south);
\draw[-] (E2.north)  -- (A2.south);
\draw[-] (E3.north)  -- (A3.south);
\draw[-] (E4.north)  -- (A4.south);
\draw[-] (E5.north)  -- (A5.south);
\draw[-] (E6.north)  -- (A6.south);
\draw[-] (E7.north)  -- (A7.south);
\draw[-] (E8.north)  -- (A8.south);

\draw[-] (E1.north)  -- (A2.south);
\draw[-] (E2.north)  -- (A1.south);
\draw[-] (E3.north)  -- (A4.south);
\draw[-] (E4.north)  -- (A3.south);
\draw[-] (E5.north)  -- (A6.south);
\draw[-] (E6.north)  -- (A5.south);
\draw[-] (E7.north)  -- (A8.south);
\draw[-] (E8.north)  -- (A7.south);

\draw[-] (A1.north)  -- (C1.south);
\draw[-] (A1.north)  -- (C3.south);
\draw[-] (A2.north)  -- (C2.south);
\draw[-] (A2.north)  -- (C4.south);
\draw[-] (A3.north)  -- (C1.south);
\draw[-] (A3.north)  -- (C3.south);
\draw[-] (A4.north)  -- (C2.south);
\draw[-] (A4.north)  -- (C4.south);
\draw[-] (A5.north)  -- (C1.south);
\draw[-] (A5.north)  -- (C3.south);
\draw[-] (A6.north)  -- (C2.south);
\draw[-] (A6.north)  -- (C4.south);
\draw[-] (A7.north)  -- (C1.south);
\draw[-] (A7.north)  -- (C3.south);
\draw[-] (A8.north)  -- (C2.south);
\draw[-] (A8.north)  -- (C4.south);

\end{tikzpicture}

\caption{The network topology modelled in this paper}

\end{figure}